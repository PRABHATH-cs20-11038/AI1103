\documentclass[journal,12pt,twocolumn]{IEEEtran}

\usepackage{setspace}
\usepackage{textcomp,gensymb}
\singlespacing
\usepackage[cmex10]{amsmath}
\usepackage{enumerate}
\usepackage{amsmath}

\usepackage{amsthm}

\usepackage{mathrsfs}
\usepackage{txfonts}
\usepackage{stfloats}
\usepackage{bm}
\usepackage{cite}
\usepackage{cases}
\usepackage[caption=false]{subfig}

\usepackage{longtable}
\usepackage{multirow}
\usepackage{multicol}
\usepackage{chngcntr}



\usepackage{mathtools}
\usepackage{steinmetz}
\usepackage{tikz}


\usepackage{verbatim}
\usepackage{tfrupee}
\usepackage[breaklinks=true]{hyperref}
\usepackage{graphicx}
\usepackage{tkz-euclide}
\usepackage[rightcaption]{sidecap}


\usepackage{graphicx} 
\graphicspath{ {images/} }


\usetikzlibrary{calc,math}
\usepackage{listings}
    \usepackage{color}                                            %%
    \usepackage{array}                                            %%
    \usepackage{longtable}                                        %%
    \usepackage{calc}                                             %%
    \usepackage{multirow}                                         %%
    \usepackage{hhline}                                           %%
    \usepackage{ifthen}                                           %%
    \usepackage{lscape}     


\DeclareMathOperator*{\Res}{Res}
\DeclareUnicodeCharacter{2212}{-}

\renewcommand\thesection{\arabic{section}}
\renewcommand\thesubsection{\thesection.\arabic{subsection}}
\renewcommand\thesubsubsection{\thesubsection.\arabic{sub-subsection}}

\renewcommand\thesectiondis{\arabic{section}}
\renewcommand\thesubsectiondis{\thesectiondis.\arabic{subsection}}
\renewcommand\thesubsubsectiondis{\thesubsectiondis.\arabic{sub-subsection}}


\hyphenation{op-ti-cal net-works semi-conduc-tor}
\def\inputGnumericTable{}                                 %%

\lstset{
%language=C,
frame=single, 
breaklines=true,
columns=fullflexible
}

\begin{document}


\newtheorem{theorem}{Theorem}[section]
\newtheorem{problem}{Problem}
\newtheorem{proposition}{Proposition}[section]
\newtheorem{lemma}{Lemma}[section]
\newtheorem{corollary}[theorem]{Corollary}
\newtheorem{example}{Example}[section]
\newtheorem{definition}[problem]{Definition}

\newcommand{\BEQA}{\begin{eqnarray}}
\newcommand{\EEQA}{\end{eqnarray}}
\newcommand{\define}{\stackrel{\triangle}{=}}
\bibliographystyle{IEEEtran}
\raggedbottom
\setlength{\parindent}{0pt}
\providecommand{\mbf}{\mathbf}
\providecommand{\pr}[1]{\ensuremath{\Pr\left(#1\right)}}
\providecommand{\qfunc}[1]{\ensuremath{Q\left(#1\right)}}
\providecommand{\sbrak}[1]{\ensuremath{{}\left[#1\right]}}
\providecommand{\lsbrak}[1]{\ensuremath{{}\left[#1\right.}}
\providecommand{\rsbrak}[1]{\ensuremath{{}\left.#1\right]}}
\providecommand{\brak}[1]{\ensuremath{\left(#1\right)}}
\providecommand{\lbrak}[1]{\ensuremath{\left(#1\right.}}
\providecommand{\rbrak}[1]{\ensuremath{\left.#1\right)}}
\providecommand{\cbrak}[1]{\ensuremath{\left\{#1\right\}}}
\providecommand{\lcbrak}[1]{\ensuremath{\left\{#1\right.}}
\providecommand{\rcbrak}[1]{\ensuremath{\left.#1\right\}}}
\theoremstyle{remark}
\newtheorem{rem}{Remark}
\newcommand{\sgn}{\mathop{\mathrm{sgn}}}
\providecommand{\res}[1]{\Res\displaylimits_{#1}} 
%\providecommand{\norm}[1]{\lVert#1\rVert}
\providecommand{\mtx}[1]{\mathbf{#1}}
\providecommand{\fourier}{\overset{\mathcal{F}}{ \rightleftharpoons}}
\providecommand{\hilbert}{\overset{\mathcal{H}}{ \rightleftharpoons}}
\providecommand{\system}{\overset{\mathcal{H}}{ \longleftrightarrow}}
	\newcommand{\solution}[2]{\textbf{Solution:}{#1}}
%\newcommand{\cosec}{\,\text{cosec}\,}
\providecommand{\dec}[2]{\ensuremath{\overset{#1}{\underset{#2}{\gtrless}}}}
\newcommand{\myvec}[1]{\ensuremath{\begin{pmatrix}#1\end{pmatrix}}}
\newcommand{\mydet}[1]{\ensuremath{\begin{vmatrix}#1\end{vmatrix}}}
\numberwithin{equation}{subsection}
\makeatletter
\@addtoreset{figure}{problem}
\makeatother
\let\StandardTheFigure\thefigure
\let\vec\mathbf
\renewcommand{\thefigure}{\theproblem}
\def\putbox#1#2#3{\makebox[0in][l]{\makebox[#1][l]{}\raisebox{\baselineskip}[0in][0in]{\raisebox{#2}[0in][0in]{#3}}}}
     \def\rightbox#1{\makebox[0in][r]{#1}}
     \def\centbox#1{\makebox[0in]{#1}}
     \def\topbox#1{\raisebox{-\baselineskip}[0in][0in]{#1}}
     \def\midbox#1{\raisebox{-0.5\baselineskip}[0in][0in]{#1}}
\title{Assignment 8}
\author{Prabhath Chellingi - CS20BTECH11038}
\maketitle
\newpage
\bigskip
\renewcommand{\thefigure}{\theenumi}
\renewcommand{\thetable}{\theenumi}

Download all python codes from 
\begin{lstlisting}

\end{lstlisting}

and latex-tikz codes from
\begin{lstlisting}

\end{lstlisting}

\section{Problem}

$(CSIR-UGC-NET\_EXAM(June-2013), Q.60)$ Consider the quadratic equation $x^2+2U x+V=0$ where $U$ and $V$ are chosen independently and randomly from $\{1,2,3\}$ with equal probabilities. Then probability that the equation has both roots real
\begin{multicols}{4}
\begin{enumerate}
    \item $\frac{2}{3}$
    \item $\frac{1}{2}$
    \item $\frac{7}{9}$
    \item $\frac{1}{3}$
\end{enumerate}
\end{multicols}

\section{Solution}

Let $U\in\{1.2,3\}$ and $V\in\{1,2,3\}$
\begin{table}[h!]
\centering
\caption{Probability of selecting values for $U$}
\resizebox{\columnwidth}{!}{
  \begin{tabular}{||c|c|c|c||}
    \hline
    $k$ & $1$ & $2$ & $3$\\
    \hline
    \hline
    $\pr{U=k}$ & $1/3$ & $1/3$ & $1/3$\\
    \hline
  \end{tabular}
}
\label{Table1}
\end{table}
\begin{table}[h!]
\centering
\caption{Probability of selecting values for $V$}
\resizebox{\columnwidth}{!}{
  \begin{tabular}{||c|c|c|c||}
    \hline
    $k$ & $1$ & $2$ & $3$\\
    \hline
    \hline
    $\pr{V=k}$ & $1/3$ & $1/3$ & $1/3$\\
    \hline
  \end{tabular}
}
\label{Table2}
\end{table}

For $x^2+2U x+V=0$ to have real roots,
\begin{align}
    b^2-4ac\geq0\\
    \brak{2U}^2-4\brak{1}\brak{V}\geq0\\
    U^2\geq V
\end{align}
The possible pairs of $\brak{U,V}$ for having real roots are
\begin{multline}
    \brak{U,V}=\{\brak{1,1}, \brak{2,1}, \brak{2,2}, \brak{2,3},\\ \brak{3,1}, \brak{3,2}, \brak{3,3}\}
\end{multline}
Let $\pr{T}$ be total probability,
\begin{multline}
    \pr{T}=\pr{\brak{U=1}\brak{V=1}} \\+ \pr{\brak{U=2}\brak{V=1}} + \pr{\brak{U=2}\brak{V=2}} \\+ \pr{\brak{U=2}\brak{V=3}} + \pr{\brak{U=3}\brak{V=1}} \\+ \pr{\brak{U=3}\brak{V=2}} + \pr{\brak{U=3}\brak{V=3}}
\end{multline}
as $U$ and $V$ are independent variables,
\begin{align}
    \pr{(U)(V)}=\pr{U}.\pr{V}
\end{align}
\begin{multline}
    \pr{T}=\pr{U=1}.\pr{V=1} \\+ \pr{U=2}.\pr{V=1} + \pr{U=2}.\pr{V=2} \\+ \pr{U=2}.\pr{V=3} + \pr{U=3}.\pr{V=1} \\+ \pr{U=3}.\pr{V=2} + \pr{U=3}.\pr{V=3}
\end{multline}
\begin{align}
    \pr{T}=7\times\brak{\frac{1}{3}}\brak{\frac{1}{3}}\\
    \pr{T}=\frac{7}{9}
\end{align}
Hence, Option 3 is correct.
\begin{lstlisting}
Probability -
      actual: 0.7778
      simulated:0.7769
\end{lstlisting}

\end{document}