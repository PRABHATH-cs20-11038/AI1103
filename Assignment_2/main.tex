\documentclass[journal,12pt,twocolumn]{IEEEtran}

\usepackage{setspace}
\usepackage{textcomp,gensymb}
\singlespacing
\usepackage[cmex10]{amsmath}
\usepackage{enumerate}
\usepackage{amsmath}

\usepackage{amsthm}

\usepackage{mathrsfs}
\usepackage{txfonts}
\usepackage{stfloats}
\usepackage{bm}
\usepackage{cite}
\usepackage{cases}
\usepackage[caption=false]{subfig}

\usepackage{longtable}
\usepackage{multirow}



\usepackage{mathtools}
\usepackage{steinmetz}
\usepackage{tikz}


\usepackage{verbatim}
\usepackage{tfrupee}
\usepackage[breaklinks=true]{hyperref}
\usepackage{graphicx}
\usepackage{tkz-euclide}
\usepackage[rightcaption]{sidecap}


\usepackage{graphicx} 
\graphicspath{ {images/} }


\usetikzlibrary{calc,math}
\usepackage{listings}
    \usepackage{color}                                            %%
    \usepackage{array}                                            %%
    \usepackage{longtable}                                        %%
    \usepackage{calc}                                             %%
    \usepackage{multirow}                                         %%
    \usepackage{hhline}                                           %%
    \usepackage{ifthen}                                           %%
    \usepackage{lscape}     


\DeclareMathOperator*{\Res}{Res}
\DeclareUnicodeCharacter{2212}{-}

\renewcommand\thesection{\arabic{section}}
\renewcommand\thesubsection{\thesection.\arabic{subsection}}
\renewcommand\thesubsubsection{\thesubsection.\arabic{sub-subsection}}

\renewcommand\thesectiondis{\arabic{section}}
\renewcommand\thesubsectiondis{\thesectiondis.\arabic{subsection}}
\renewcommand\thesubsubsectiondis{\thesubsectiondis.\arabic{sub-subsection}}


\hyphenation{op-ti-cal net-works semi-conduc-tor}
\def\inputGnumericTable{}                                 %%

\lstset{
%language=C,
frame=single, 
breaklines=true,
columns=fullflexible
}

\begin{document}


\newtheorem{theorem}{Theorem}[section]
\newtheorem{problem}{Problem}
\newtheorem{proposition}{Proposition}[section]
\newtheorem{lemma}{Lemma}[section]
\newtheorem{corollary}[theorem]{Corollary}
\newtheorem{example}{Example}[section]
\newtheorem{definition}[problem]{Definition}

\newcommand{\BEQA}{\begin{eqnarray}}
\newcommand{\EEQA}{\end{eqnarray}}
\newcommand{\define}{\stackrel{\triangle}{=}}
\bibliographystyle{IEEEtran}
\raggedbottom
\setlength{\parindent}{0pt}
\providecommand{\mbf}{\mathbf}
\providecommand{\pr}[1]{\ensuremath{\Pr\left(#1\right)}}
\providecommand{\qfunc}[1]{\ensuremath{Q\left(#1\right)}}
\providecommand{\sbrak}[1]{\ensuremath{{}\left[#1\right]}}
\providecommand{\lsbrak}[1]{\ensuremath{{}\left[#1\right.}}
\providecommand{\rsbrak}[1]{\ensuremath{{}\left.#1\right]}}
\providecommand{\brak}[1]{\ensuremath{\left(#1\right)}}
\providecommand{\lbrak}[1]{\ensuremath{\left(#1\right.}}
\providecommand{\rbrak}[1]{\ensuremath{\left.#1\right)}}
\providecommand{\cbrak}[1]{\ensuremath{\left\{#1\right\}}}
\providecommand{\lcbrak}[1]{\ensuremath{\left\{#1\right.}}
\providecommand{\rcbrak}[1]{\ensuremath{\left.#1\right\}}}
\theoremstyle{remark}
\newtheorem{rem}{Remark}
\newcommand{\sgn}{\mathop{\mathrm{sgn}}}
\providecommand{\res}[1]{\Res\displaylimits_{#1}} 
%\providecommand{\norm}[1]{\lVert#1\rVert}
\providecommand{\mtx}[1]{\mathbf{#1}}
\providecommand{\fourier}{\overset{\mathcal{F}}{ \rightleftharpoons}}
\providecommand{\hilbert}{\overset{\mathcal{H}}{ \rightleftharpoons}}
\providecommand{\system}{\overset{\mathcal{H}}{ \longleftrightarrow}}
	\newcommand{\solution}[2]{\textbf{Solution:}{#1}}
%\newcommand{\cosec}{\,\text{cosec}\,}
\providecommand{\dec}[2]{\ensuremath{\overset{#1}{\underset{#2}{\gtrless}}}}
\newcommand{\myvec}[1]{\ensuremath{\begin{pmatrix}#1\end{pmatrix}}}
\newcommand{\mydet}[1]{\ensuremath{\begin{vmatrix}#1\end{vmatrix}}}
\numberwithin{equation}{subsection}
\makeatletter
\@addtoreset{figure}{problem}
\makeatother
\let\StandardTheFigure\thefigure
\let\vec\mathbf
\renewcommand{\thefigure}{\theproblem}
\def\putbox#1#2#3{\makebox[0in][l]{\makebox[#1][l]{}\raisebox{\baselineskip}[0in][0in]{\raisebox{#2}[0in][0in]{#3}}}}
     \def\rightbox#1{\makebox[0in][r]{#1}}
     \def\centbox#1{\makebox[0in]{#1}}
     \def\topbox#1{\raisebox{-\baselineskip}[0in][0in]{#1}}
     \def\midbox#1{\raisebox{-0.5\baselineskip}[0in][0in]{#1}}
\vspace{3cm}
\title{Assignment 2}
\author{Prabhath Chellingi - CS20BTECH11038}
\maketitle
\newpage
\bigskip
\renewcommand{\thefigure}{\theenumi}
\renewcommand{\thetable}{\theenumi}

Download all python codes from 
\begin{lstlisting}
https://github.com/PRABHATH-cs20-11038/Assignment_1/tree/main/Assignment_2
\end{lstlisting}

and latex-tikz codes from
\begin{lstlisting}
https://github.com/PRABHATH-cs20-11038/Assignment_1/tree/main/Assignment_2
\end{lstlisting}

\section{Problem}

$(GATE-EC-66)$ Consider two identical boxes $B_1$ and $B_2$, where the box $B_i(i = 1, 2)$ contains $i + 2$ red and $5−i−1$ white balls. A fair die is cast. Let the number of dots shown on the top face of the die be $N$. If $N$ is even or $5$, then two balls are drawn with replacement from the box $B_1$, otherwise, two balls are drawn with replacement from the box $B_2$. The probability that the two drawn balls are of different colours is

\begin{enumerate}[(A)]
    \item $\frac{7}{25}$ \\
    
    \item $\frac{9}{25}$ \\
    
    \item $\frac{12}{25}$ \\
    
    \item $\frac{16}{25}$
\end{enumerate}

\section{Solution}

Let $X \in \{1,2,3,4,5,6\}$ be the random variables representing the outcome for a die. The $PDF$ of die is

\begin{align}
    \pr{X=N} =
    \begin{cases}
    \frac{1}{6} & 1 \leq N \leq 6\\
    0 & otherwise
    \end{cases}
\end{align}
where $N$ = the number on top face of the die.

\vspace{0.2in}

As we know
\begin{align}\label{eq1}
    \pr{X=m}.\pr{X=n}=0
\end{align}
for all $m,n \in \{1,2,3,4,5,6\}$ as a single die cannot show more than one outcome at a roll.
\vspace{0.2in}

Let $Y \in \{0, 1\}$ represent the die where $1$ denotes the die with outcome $N = \{2,4,5,6\}$ and $0$ denotes the remaining.

\begin{multline}
    \pr{Y=1}=\\
    \pr{(X=2)+(X=4)+(X=5)+(X=6)}
\end{multline}
\newline

by using \textit{Boolean logic} and equation \eqref{eq1}, we get,
\begin{multline}
    \pr{Y=1}=\\
    \pr{X=2}+\pr{X=4}+\pr{X=5}+\pr{X=6}
\end{multline}

\begin{align}\label{eq2}
    \begin{split}
        \pr{Y=1}&=\frac{1}{6}+\frac{1}{6}+\frac{1}{6}+\frac{1}{6}\\
        &=\frac{2}{3}
    \end{split}
\end{align}

\begin{align}\label{eq3}
    \begin{split}
        \pr{Y=0}&=1-\pr{Y=1}\\
        &=\frac{1}{3}
    \end{split}
\end{align}
\newline

let $C \in \{0,1\}$ where 0 denotes to red balls and 1 denotes to white balls.
\newline

There are two boxes $B_i(i = 1, 2)$ containing $i + 2$ red and $5−i−1$ white balls.

\begin{enumerate}[(a)]
\item
number of red balls$(C=0)$ in $B_1$,
\begin{align}
    n(C=0|B_1)=(1)+2\\
    n(C=0|B_1)=3
\end{align}
\item
number of white balls$(C=1)$ in $B_1$,
\begin{align}
    n(C=1|B_1)=5-(1)-1\\
    n(C=1|B_1)=3
\end{align}
\end{enumerate}

number of balls in box $B_1$,
\begin{align}
    n(C|B_1)=n(C=0|B_1)+ n(C=1|B_1)\\
    n(C|B_1)=6
\end{align}

\begin{enumerate}[(a)]
\item 
probability of selecting red ball from box $B_1$ is,
\begin{align}
\begin{split}
    \pr{C=0|B_1}&=\frac{n(C=0|B_1)}{n(C=0|B_1)+ n(C=1|B_1)}\\
    &=\frac{1}{2}
\end{split}
\end{align}
\item
probability of selecting white ball from box $B_1$ is,
\begin{align}
\begin{split}
    \pr{C=1|B_1}&=\frac{n(C=1|B_1)}{n(C=0|B_1)+ n(C=1|B_1)}\\
    &=\frac{1}{2}
\end{split}
\end{align}
\end{enumerate}

\begin{enumerate}[(a)]
\item
number of red balls$(C=0)$ in $B_2$,
\begin{align}
    n(C=0|B_2)=(2)+2\\
    n(C=0|B_2)=4
\end{align}
\item
number of white balls$(C=1)$ in $B_2$,
\begin{align}
    n(C=1|B_1)=5-(2)-1\\
    n(C=1|B_1)=2
\end{align}
\end{enumerate}

number of balls in box $B_2$,
\begin{align}
    n(C|B_2)=n(C=0|B_2)+ n(C=1|B_2)\\
    n(C|B_1)=6
\end{align}

\begin{enumerate}[(a)]
\item
probability of selecting red ball from box $B_2$ is,
\begin{align}
\begin{split}
    \pr{C=0|B_2}&=\frac{n(C=0|B_2)}{n(C=0|B_2)+ n(C=1|B_2)}\\
    &=\frac{2}{3}
\end{split}
\end{align}
\item
probability of selecting white ball from box $B_2$ is,
\begin{align}
\begin{split}
    \pr{C=1|B_2}&=\frac{n(C=1|B_2)}{n(C=0|B_2)+ n(C=1|B_2)}\\
    &=\frac{1}{3}
\end{split}
\end{align}
\end{enumerate}

Given the probability of selecting box $B_1$ is same as $\pr{Y=1}$, from equation \eqref{eq2},
\begin{align}
    \pr{B_1}=\pr{Y=1}
\end{align}

\begin{align}\label{eq4}
    \pr{B_1}=\frac{2}{3}
\end{align}

Given the probability of selecting box $B_2$ is same as $\pr{Y=0}$, from equation \eqref{eq3},
\begin{align}
    \pr{B_2}=\pr{Y=0}
\end{align}

\begin{align}\label{eq5}
    \pr{B_2}=\frac{1}{3}
\end{align}

No of ways of selecting two different coloured balls is 
\begin{enumerate}[(1)]
    \item $(B_1,C=0,C=1)$
    \item $(B_1,C=1,C=0)$
    \item $(B_2,C=0,C=1)$
    \item $(B_2,C=1,C=0)$
\end{enumerate}

\vspace{0.2in}

The probability of second ball is not effected because the second ball is chose after replacement.
\vspace{0.2in}

Let $\pr{(C=0,C=1)|B_1}$ be probability of selecting two different coloured balls from Bag $B_1$.

Probability of selecting two different colored balls from bag $B_1$ is, (by using Boolean logic)
\begin{multline}
    \pr{(C=0,C=1)|B_1}=\\
    \pr{C=0|B_1}.\pr{C=1|B_1}+\\
    \pr{C=1|B_1}.\pr{C=0|B_1}
\end{multline}

\begin{align}
        \pr{(C=0,C=1)|B_1}=\frac{1}{2} \times\frac{1}{2}+\frac{1}{2}\times\frac{1}{2}
\end{align}

\begin{align}\label{eq6}
    \pr{(C=0,C=1)|B_1}=\frac{1}{2}
\end{align}

Let $\pr{(C=0,C=1)|B_2}$ be probability of selecting two different coloured balls from Bag $B_2$.

Probability of selecting two different colored balls from bag $B_2$ is, (by using Boolean logic)

\begin{multline}
    \pr{(C=0,C=1)|B_2}=\\
    \pr{C=0|B_2}.\pr{C=1|B_2}+\\
    \pr{C=1|B_2}.\pr{C=0|B_2}
\end{multline}

\begin{align}
        \pr{(C=0,C=1)|B_2}=\frac{2}{3}\times\frac{1}{3}+\frac{1}{3}\times\frac{2}{3}
\end{align}

\begin{align}\label{eq7}
    \pr{(C=0,C=1)|B_2}=\frac{4}{9}
\end{align}
\newline

let probability of selecting two different coloured balls be $\pr{T}$ 
\newline

Now we can obtain $\pr{T}$ by using \textit{conditional probability} 
\begin{align}\label{eq8}
    \pr{E F}=\pr{E|F}.\pr{F}
\end{align}

\begin{multline}
    \pr{T}=\\
    \pr{((C=0,C=1)|B_1)(B_1)}+\\
    pr{((C=0,C=1)|B_2)(B_2)}
\end{multline}

from equation \eqref{eq8}
\begin{multline}
    \pr{T}=\\
    \pr{(C=0,C=1)|B_1}.\pr{B_1}+\\
    \pr{(C=0,C=1)|B_2}.\pr{B_2}
\end{multline}

from \eqref{eq4}, \eqref{eq5}, \eqref{eq6}, \eqref{eq7},
\begin{align}
    \pr{T}=\brak{\frac{1}{2}}\brak{\frac{2}{3}}+\brak{\frac{4}{9}}\brak{\frac{1}{3}}\\
    \pr{T}=\frac{13}{27}
\end{align}

\vspace{0.2in}

Hence, the probability of selecting two different coloured balls from the bags is \[\frac{13}{27}\]

\begin{lstlisting}
Probability-
    simulation: 0.48015,
    actual: 0.48148148148148145
\end{lstlisting}

\end{document}
