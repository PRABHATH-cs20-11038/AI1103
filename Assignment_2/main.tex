\documentclass[journal,12pt,twocolumn]{IEEEtran}

\usepackage{setspace}
\usepackage{textcomp,gensymb}
\singlespacing
\usepackage[cmex10]{amsmath}
\usepackage{enumerate}
\usepackage{amsmath}

\usepackage{amsthm}

\usepackage{mathrsfs}
\usepackage{txfonts}
\usepackage{stfloats}
\usepackage{bm}
\usepackage{cite}
\usepackage{cases}
\usepackage[caption=false]{subfig}

\usepackage{longtable}
\usepackage{multirow}



\usepackage{mathtools}
\usepackage{steinmetz}
\usepackage{tikz}


\usepackage{verbatim}
\usepackage{tfrupee}
\usepackage[breaklinks=true]{hyperref}
\usepackage{graphicx}
\usepackage{tkz-euclide}
\usepackage[rightcaption]{sidecap}


\usepackage{graphicx} 
\graphicspath{ {images/} }


\usetikzlibrary{calc,math}
\usepackage{listings}
    \usepackage{color}                                            %%
    \usepackage{array}                                            %%
    \usepackage{longtable}                                        %%
    \usepackage{calc}                                             %%
    \usepackage{multirow}                                         %%
    \usepackage{hhline}                                           %%
    \usepackage{ifthen}                                           %%
    \usepackage{lscape}     


\DeclareMathOperator*{\Res}{Res}
\DeclareUnicodeCharacter{2212}{-}

\renewcommand\thesection{\arabic{section}}
\renewcommand\thesubsection{\thesection.\arabic{subsection}}
\renewcommand\thesubsubsection{\thesubsection.\arabic{sub-subsection}}

\renewcommand\thesectiondis{\arabic{section}}
\renewcommand\thesubsectiondis{\thesectiondis.\arabic{subsection}}
\renewcommand\thesubsubsectiondis{\thesubsectiondis.\arabic{sub-subsection}}


\hyphenation{op-ti-cal net-works semi-conduc-tor}
\def\inputGnumericTable{}                                 %%

\lstset{
%language=C,
frame=single, 
breaklines=true,
columns=fullflexible
}

\begin{document}


\newtheorem{theorem}{Theorem}[section]
\newtheorem{problem}{Problem}
\newtheorem{proposition}{Proposition}[section]
\newtheorem{lemma}{Lemma}[section]
\newtheorem{corollary}[theorem]{Corollary}
\newtheorem{example}{Example}[section]
\newtheorem{definition}[problem]{Definition}

\newcommand{\BEQA}{\begin{eqnarray}}
\newcommand{\EEQA}{\end{eqnarray}}
\newcommand{\define}{\stackrel{\triangle}{=}}
\bibliographystyle{IEEEtran}
\raggedbottom
\setlength{\parindent}{0pt}
\providecommand{\mbf}{\mathbf}
\providecommand{\pr}[1]{\ensuremath{\Pr\left(#1\right)}}
\providecommand{\qfunc}[1]{\ensuremath{Q\left(#1\right)}}
\providecommand{\sbrak}[1]{\ensuremath{{}\left[#1\right]}}
\providecommand{\lsbrak}[1]{\ensuremath{{}\left[#1\right.}}
\providecommand{\rsbrak}[1]{\ensuremath{{}\left.#1\right]}}
\providecommand{\brak}[1]{\ensuremath{\left(#1\right)}}
\providecommand{\lbrak}[1]{\ensuremath{\left(#1\right.}}
\providecommand{\rbrak}[1]{\ensuremath{\left.#1\right)}}
\providecommand{\cbrak}[1]{\ensuremath{\left\{#1\right\}}}
\providecommand{\lcbrak}[1]{\ensuremath{\left\{#1\right.}}
\providecommand{\rcbrak}[1]{\ensuremath{\left.#1\right\}}}
\theoremstyle{remark}
\newtheorem{rem}{Remark}
\newcommand{\sgn}{\mathop{\mathrm{sgn}}}
\providecommand{\res}[1]{\Res\displaylimits_{#1}} 
%\providecommand{\norm}[1]{\lVert#1\rVert}
\providecommand{\mtx}[1]{\mathbf{#1}}
\providecommand{\fourier}{\overset{\mathcal{F}}{ \rightleftharpoons}}
\providecommand{\hilbert}{\overset{\mathcal{H}}{ \rightleftharpoons}}
\providecommand{\system}{\overset{\mathcal{H}}{ \longleftrightarrow}}
	\newcommand{\solution}[2]{\textbf{Solution:}{#1}}
%\newcommand{\cosec}{\,\text{cosec}\,}
\providecommand{\dec}[2]{\ensuremath{\overset{#1}{\underset{#2}{\gtrless}}}}
\newcommand{\myvec}[1]{\ensuremath{\begin{pmatrix}#1\end{pmatrix}}}
\newcommand{\mydet}[1]{\ensuremath{\begin{vmatrix}#1\end{vmatrix}}}
\numberwithin{equation}{subsection}
\makeatletter
\@addtoreset{figure}{problem}
\makeatother
\let\StandardTheFigure\thefigure
\let\vec\mathbf
\renewcommand{\thefigure}{\theproblem}
\def\putbox#1#2#3{\makebox[0in][l]{\makebox[#1][l]{}\raisebox{\baselineskip}[0in][0in]{\raisebox{#2}[0in][0in]{#3}}}}
     \def\rightbox#1{\makebox[0in][r]{#1}}
     \def\centbox#1{\makebox[0in]{#1}}
     \def\topbox#1{\raisebox{-\baselineskip}[0in][0in]{#1}}
     \def\midbox#1{\raisebox{-0.5\baselineskip}[0in][0in]{#1}}
\title{Assignment 2}
\author{Prabhath Chellingi - CS20BTECH11038}
\maketitle
\newpage
\bigskip
\renewcommand{\thefigure}{\theenumi}
\renewcommand{\thetable}{\theenumi}

Download all python codes from 
\begin{lstlisting}
https://github.com/PRABHATH-cs20-11038/AI1103/tree/main/Assignment_2
\end{lstlisting}

and latex-tikz codes from
\begin{lstlisting}
https://github.com/PRABHATH-cs20-11038/AI1103/tree/main/Assignment_2
\end{lstlisting}

\section{Problem}

$(GATE-EC-66)$ Consider two identical boxes $B_1$ and $B_2$, where the box $B_i(i = 1, 2)$ contains $i + 2$ red and $5−i−1$ white balls. A fair die is cast. Let the number of dots shown on the top face of the die be $N$. If $N$ is even or $5$, then two balls are drawn with replacement from the box $B_1$, otherwise, two balls are drawn with replacement from the box $B_2$. The probability that the two drawn balls are of different colours is

\section{Solution}

Let $X \in \{1,2,3,4,5,6\}$ be the random variables of a die,
\begin{align}
    \pr{X=N} =
    \begin{cases}
    \frac{1}{6} & 1 \leq N \leq 6\\
    0 & otherwise
    \end{cases}
\end{align}

\begin{align}\label{eq1}
    \pr{X=m}.\pr{X=n}=0
\end{align}
for all $m,n \in \{1,2,3,4,5,6\}$ as a single die cannot show more than one outcome at a roll.

\vspace{0.1in}

Let $Y \in \{0, 1\}$ represent the die where,

$1$ $\implies$ the die with outcome $N = \{ 2, 4, 5, 6\}$,

$0$ $\implies$ $N = \{ 1, 3\}$.
\begin{multline}
    \pr{Y=1}=\\
    \pr{(X=2)+(X=4)+(X=5)+(X=6)}
\end{multline}

by using \textit{Boolean logic} and equation \eqref{eq1},
\begin{align}
    \pr{Y=1}=\frac{2}{3}\\
    \pr{Y=0}=1-\pr{Y=1}=\frac{1}{3}
\end{align}

\begin{enumerate}[(i)]
\item
probability of selecting box $B_1$,
\begin{align}\label{eq2}
    \pr{B_1}=\pr{Y=1}=\frac{2}{3}
\end{align}
\item
probability of selecting box $B_2$,
\begin{align}\label{eq3}
    \pr{B_2}=\pr{Y=0}=\frac{1}{3}
\end{align}
\end{enumerate}

let $C \in \{0,1\}$ where, 

$0$ $\implies$ red balls,

$1$ $\implies$ white balls.

\begin{table}[h!]
\centering
\caption{Table of number of balls}
\resizebox{.5\textwidth}{!}{
  \begin{tabular}{|c|m{3cm}|m{3cm}|c|}
  \hline
    Box & No. of red balls $(i+2)$ & No. of white balls $(5-i-1)$ & Total balls\\
    \hline
    $B_1$ & $n(C=0|B_1)=3$ & $n(C=1|B_1)=3$ & $n(C|B_1)=6$\\
    \hline
    $B_2$ & $n(C=0|B_2)=4$ & $n(C=1|B_2)=2$ & $n(C|B_2)=6$\\
    \hline
  \end{tabular}
}
\label{table1}
\end{table}


\begin{table}[h!]
\centering
\caption{Table of probability of taking balls from each box}
\resizebox{.5\textwidth}{!}{
  \begin{tabular}{|c|m{4cm}|m{4cm}|}
  \hline
    Box & Probability of taking red ball & Probability of taking white ball\\
    \hline
    $B_1$ & $\pr{C=0|B_1}=1/2$ & $\pr{C=1|B_1}=1/2$\\
    \hline
    $B_2$ & $\pr{C=0|B_2}=2/3$ & $\pr{C=1|B_2} = 1/3$\\
    \hline
  \end{tabular}
}
\label{Table2}
\end{table}

The probability of picking second ball is not effected by picking first ball because the second ball is chose after replacement.

\vspace{0.1in}

Selecting two balls with replacement is a \textit{Bernoulli distribution} of $2$ trails,
\begin{table}
\centering
\caption{Table of no. of ways of selecting two different coloured balls}
\resizebox{.5\textwidth}{!}{
  \begin{tabular}{|c|c|c|}
    \hline
    Cases & Trail 1 & Trail 2\\
    \hline
    $(B_1,C=0,C=1)$ & $\pr{C=0|B_1}$ & $\pr{C=1|B_1}$\\
    \hline
    $(B_1,C=1,C=0)$ & $\pr{C=1|B_1}$ & $\pr{C=0|B_1}$\\
    \hline
    $(B_2,C=0,C=1)$ & $\pr{C=0|B_2}$ & $\pr{C=1|B_2}$\\
    \hline
    $(B_2,C=1,C=0)$ & $\pr{C=1|B_2}$ & $\pr{C=0|B_2}$\\
    \hline
  \end{tabular}
  \label{Table3}
}
\end{table}

\begin{table}
\centering
\caption{Table of variables description}
\resizebox{.5\textwidth}{!}{
  \begin{tabular}{|c|m{5cm}|}
    \hline
    Variables & Description\\
    \hline
    $\pr{(C=0,C=1)|B_1}$ & Probability of selecting two different coloured balls from box $B_1$\\
    \hline
    $\pr{(C=0,C=1)|B_2}$ & Probability of selecting two different coloured balls from box $B_2$\\
    \hline
    $\pr{T}$ & Total probability of selecting two different coloured balls\\
    \hline
  \end{tabular}
}
\label{Table4}
\end{table}

\begin{enumerate}[(i)]
\item
\begin{multline}
    \pr{(C=0,C=1)|B_1}=\\\pr{C=0|B_1}.\pr{C=1|B_1}\\+\pr{C=1|B_1}.\pr{C=0|B_1}
\end{multline}
\begin{align}
    \pr{(C=0,C=1)|B_1}=\frac{1}{2}
\end{align}
\item
\begin{multline}
    \pr{(C=0,C=1)|B_2}=\\\pr{C=0|B_2}.\pr{C=1|B_2}\\+\pr{C=1|B_2}.\pr{C=0|B_2}
\end{multline}
\begin{align}
    \pr{(C=0,C=1)|B_1}=\frac{4}{9}
\end{align}
\end{enumerate}

by using \textit{Bayes theorem},
\begin{multline}
    \pr{T}=\\
    \pr{(C=0,C=1)|B_1}.\pr{B_1}+\\
    \pr{(C=0,C=1)|B_2}.\pr{B_2}
\end{multline}

\begin{align}
    \pr{T}=\brak{\frac{1}{2}}\brak{\frac{2}{3}}+\brak{\frac{4}{9}}\brak{\frac{1}{3}}
\end{align}

Hence, the probability of selecting two different coloured balls from the boxes is

\begin{align}
    \pr{T}=\frac{13}{27}
\end{align}

\begin{lstlisting}
Probability-
    simulation: 0.48015,
    actual: 0.48148148148148145
\end{lstlisting}

\end{document}
