\documentclass[journal,12pt,twocolumn]{IEEEtran}

\usepackage{setspace}
\usepackage{textcomp,gensymb}
\singlespacing
\usepackage[cmex10]{amsmath}
\usepackage{enumerate}
\usepackage{amsmath}

\usepackage{amsthm}

\usepackage{mathrsfs}
\usepackage{txfonts}
\usepackage{stfloats}
\usepackage{bm}
\usepackage{cite}
\usepackage{cases}
\usepackage[caption=false]{subfig}

\usepackage{longtable}
\usepackage{multirow}

\usepackage{mathtools}
\usepackage{steinmetz}
\usepackage{tikz}

\usepackage{verbatim}
\usepackage{tfrupee}
\usepackage[breaklinks=true]{hyperref}
\usepackage{graphicx}
\usepackage{tkz-euclide}
\usepackage[rightcaption]{sidecap}

\usepackage{graphicx} 
\graphicspath{ {images/} }

\usetikzlibrary{calc,math}
\usepackage{listings}
    \usepackage{color}                                            %%
    \usepackage{array}                                            %%
    \usepackage{longtable}                                        %%
    \usepackage{calc}                                             %%
    \usepackage{multirow}                                         %%
    \usepackage{hhline}                                           %%
    \usepackage{ifthen}                                           %%
    \usepackage{lscape}     

\DeclareMathOperator*{\Res}{Res}
\DeclareUnicodeCharacter{2212}{-}

\renewcommand\thesection{\arabic{section}}
\renewcommand\thesubsection{\thesection.\arabic{subsection}}
\renewcommand\thesubsubsection{\thesubsection.\arabic{sub-subsection}}

\renewcommand\thesectiondis{\arabic{section}}
\renewcommand\thesubsectiondis{\thesectiondis.\arabic{subsection}}
\renewcommand\thesubsubsectiondis{\thesubsectiondis.\arabic{sub-subsection}}

\hyphenation{op-ti-cal net-works semi-conduc-tor}
\def\inputGnumericTable{}                                 %%

\lstset{
%language=C,
frame=single, 
breaklines=true,
columns=fullflexible
}

\begin{document}

\newtheorem{theorem}{Theorem}[section]
\newtheorem{problem}{Problem}
\newtheorem{proposition}{Proposition}[section]
\newtheorem{lemma}{Lemma}[section]
\newtheorem{corollary}[theorem]{Corollary}
\newtheorem{example}{Example}[section]
\newtheorem{definition}[problem]{Definition}

\newcommand{\BEQA}{\begin{eqnarray}}
\newcommand{\EEQA}{\end{eqnarray}}
\newcommand{\define}{\stackrel{\triangle}{=}}
\bibliographystyle{IEEEtran}
\raggedbottom
\setlength{\parindent}{0pt}
\providecommand{\mbf}{\mathbf}
\providecommand{\pr}[1]{\ensuremath{\Pr\left(#1\right)}}
\providecommand{\qfunc}[1]{\ensuremath{Q\left(#1\right)}}
\providecommand{\sbrak}[1]{\ensuremath{{}\left[#1\right]}}
\providecommand{\lsbrak}[1]{\ensuremath{{}\left[#1\right.}}
\providecommand{\rsbrak}[1]{\ensuremath{{}\left.#1\right]}}
\providecommand{\brak}[1]{\ensuremath{\left(#1\right)}}
\providecommand{\lbrak}[1]{\ensuremath{\left(#1\right.}}
\providecommand{\rbrak}[1]{\ensuremath{\left.#1\right)}}
\providecommand{\cbrak}[1]{\ensuremath{\left\{#1\right\}}}
\providecommand{\lcbrak}[1]{\ensuremath{\left\{#1\right.}}
\providecommand{\rcbrak}[1]{\ensuremath{\left.#1\right\}}}
\theoremstyle{remark}
\newtheorem{rem}{Remark}
\newcommand{\sgn}{\mathop{\mathrm{sgn}}}
\providecommand{\res}[1]{\Res\displaylimits_{#1}} 
%\providecommand{\norm}[1]{\lVert#1\rVert}
\providecommand{\mtx}[1]{\mathbf{#1}}
\providecommand{\fourier}{\overset{\mathcal{F}}{ \rightleftharpoons}}
\providecommand{\hilbert}{\overset{\mathcal{H}}{ \rightleftharpoons}}
\providecommand{\system}{\overset{\mathcal{H}}{ \longleftrightarrow}}
	\newcommand{\solution}[2]{\textbf{Solution:}{#1}}
%\newcommand{\cosec}{\,\text{cosec}\,}
\providecommand{\dec}[2]{\ensuremath{\overset{#1}{\underset{#2}{\gtrless}}}}
\newcommand{\myvec}[1]{\ensuremath{\begin{pmatrix}#1\end{pmatrix}}}
\newcommand{\mydet}[1]{\ensuremath{\begin{vmatrix}#1\end{vmatrix}}}
\numberwithin{equation}{subsection}
\makeatletter
\@addtoreset{figure}{problem}
\makeatother
\let\StandardTheFigure\thefigure
\let\vec\mathbf
\renewcommand{\thefigure}{\theproblem}
\def\putbox#1#2#3{\makebox[0in][l]{\makebox[#1][l]{}\raisebox{\baselineskip}[0in][0in]{\raisebox{#2}[0in][0in]{#3}}}}
     \def\rightbox#1{\makebox[0in][r]{#1}}
     \def\centbox#1{\makebox[0in]{#1}}
     \def\topbox#1{\raisebox{-\baselineskip}[0in][0in]{#1}}
     \def\midbox#1{\raisebox{-0.5\baselineskip}[0in][0in]{#1}}
\title{Assignment 7}
\author{Prabhath Chellingi - CS20BTECH11038}
\maketitle
\newpage
\bigskip
\renewcommand{\thefigure}{\theenumi}
\renewcommand{\thetable}{\theenumi}

Download latex-tikz codes from
\begin{lstlisting}
https://github.com/PRABHATH-cs20-11038/AI1103/tree/main/Assignment_7
\end{lstlisting}

\section{Problem}

$(CSIR-UGC-NET\_EXAM (Dec-2016), Q.107)$ Let $X$ be a random variable with a certain non-degenerate distribution. Then identify the correct statements
\begin{enumerate}[1.]
    \item If $X$ has an exponential distribution then $median\brak{X}<E\brak{X}$
    \item If $X$ has a uniform distribution on an interval $[a,b]$, then $E\brak{X}<median\brak{X}$
    \item If $X$ has a Binomial distribution then $V\brak{X}<E\brak{X}$
    \item If $X$ has a normal distribution, then $E\brak{X}<V\brak{X}$
\end{enumerate}

\section{Solution}
Expected value\brak{E\brak{X}}:

It is nothing but weighted average

Median\brak{median\brak{X}}:

It is the value separating the higher half from the lower half of a data sample

Variance\brak{V\brak{X}}:

It is the expectation of the squared deviation of a random variable from its mean
\begin{enumerate}
    \item Let's consider $X$ has an exponential distribution.
    \begin{align}
        X \sim Exp\brak{\lambda}
    \end{align}
    where $\lambda$ is rate parameter.
    
    Probability function of exponential distribution,
    \begin{align}
        f_X\brak{x}=
        \begin{cases}
            \lambda e^{-\lambda x} & x\geq0\\
            0 & x<0
        \end{cases}
    \end{align}
    The expected value of $X \sim Exp\brak{\lambda}$,
    \begin{align}
        E\brak{X}=\frac{1}{\lambda}
    \end{align}
    The median of $X \sim Exp\brak{\lambda}$,
    \begin{align}
        median\brak{X}=\frac{\ln{2}}{\lambda}
    \end{align}
    \begin{align}
        \ln{2}<1\\
        \frac{\ln{2}}{\lambda}<\frac{1}{\lambda}\\
         median\brak{X}<E\brak{X}
    \end{align}
    Hence, option $1$ is correct.
    
    \item Let's consider $X$ has a uniform distribution in interval $[a,b]$,
    \begin{align}
        X \sim U\brak{a,b}
    \end{align}
    where,
    $a=$ lower limit
    
    $b=$ upper limit
    
    Probability function of uniform distribution,
    \begin{align}
        f_X\brak{k}=
        \begin{cases}
            \frac{1}{b-a} & a\leq x\leq b\\
            0 & x < a, x > b
        \end{cases}
    \end{align}
    The expected value of $X \sim U\brak{a,b}$,
    \begin{align}
        E\brak{X}=\frac{1}{2}\brak{a+b}
    \end{align}
    The median of $X \sim U\brak{a,b}$,
    \begin{align}
        median\brak{X}=\frac{1}{2}\brak{a+b}
    \end{align}
    \begin{align}
        E\brak{X}=median\brak{X}
    \end{align}
    Hence, option $2$ is incorrect.
    
    \item Let's consider $X$ has a binomial distribution,
    \begin{align}
        X \sim B\brak{n,p}
    \end{align}
    where,
    $n=$ no. of trails
    
    $p=$ success parameter
    
    Probability function of binomial distribution,
    \begin{align}
        f_X\brak{k}=
        \begin{cases}
            {^n C_k}p^k(1-p)^{n-k} & 0\leq k\leq n\\
            0 & otherwise
        \end{cases}
    \end{align}
    The expected value of $X \sim B\brak{n,p}$,
    \begin{align}
        E\brak{X}=np
    \end{align}
    The variance of $X \sim B\brak{n,p}$,
    \begin{align}
        V\brak{X}=\sigma^2=n p(1-p)
    \end{align}
    \begin{align}
        1-p\leq1\\
        n p(1-p)\leq n p\\
        V\brak{X}\leq E\brak{X}
    \end{align}
    Hence, option $3$ is incorrect.
    
    \item Let's consider $X$ has a normal distribution,
    \begin{align}
        X \sim N\brak{\mu,\sigma^2}
    \end{align}
    where,
    $\mu=$ mean of distribution
    
    $\sigma^2=$ variance
    
    Probability function of normal distribution,
    \begin{align}
        f_X\brak{k}=\frac{1}{\sigma\sqrt{2\pi}}e^{-\brak{\frac{x-\mu}{2\sigma}}^2}
    \end{align}
    The expected value of $X \sim N\brak{\mu,\sigma^2}$,
    \begin{align}
        E\brak{X}=\mu
    \end{align}
    The variance of $X \sim N\brak{\mu,\sigma^2}$,
    \begin{align}
        V\brak{X}=\sigma^2
    \end{align}
    $E\brak{X}$ and $V\brak{X}$ are user defined. So, they can take any value.
    
    Hence, option $4$ is incorrect.
\end{enumerate}

\end{document}
