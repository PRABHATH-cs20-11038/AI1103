\documentclass{beamer}
\usepackage{listings}
\lstset{
%language=C,
frame=single, 
breaklines=true,
columns=fullflexible
}
\usepackage{subcaption}
\usepackage{url}
\usepackage{tikz}
\usepackage{tkz-euclide} % loads  TikZ and tkz-base
%\usetkzobj{all}
\usetikzlibrary{calc,math}
\usepackage{float}
\newcommand\norm[1]{\left\lVert#1\right\rVert}
\renewcommand{\vec}[1]{\mathbf{#1}}
\usepackage[export]{adjustbox}
\usepackage[utf8]{inputenc}
\usepackage{amsmath}
\usetheme{Boadilla}
\providecommand{\pr}[1]{\ensuremath{\Pr\left(#1\right)}}
\usepackage{mathtools}

\title{CSIR UGC NET EXAM(Dec 2016), Q.107}
\author{Prabhath Chellingi - CS20BTECH11038}
\date{}
\begin{document}

\begin{frame}
\titlepage
\end{frame}

\begin{frame}
\frametitle{Question}

\begin{block}{}
 Let $X$ be a random variable with a certain non-degenerate distribution. Then identify the correct statements
\begin{enumerate}[1.]
    \item If $X$ has an exponential distribution then $median(X)<E(X)$
    \item If $X$ has a uniform distribution on an interval $[a,b]$, then $E(X)<median(X)$
    \item If $X$ has a Binomial distribution then $V(X)<E(X)$
    \item If $X$ has a normal distribution, then $E(X)<V(X)$
\end{enumerate}
\end{block}
\end{frame}
\begin{frame}{}
    \begin{block}{Pre-requisites }
    \begin{itemize}
        \item Expected value
        \item Variance
        \item Median
        \item Exponential Distribution
        \item Uniform Distribution
        \item Binomial Distribution
        \item Normal Distribution
    \end{itemize}
    \end{block}
\end{frame}
\begin{frame}{}
\begin{block}{Expected value}
It is nothing but weighted mean

Notation - $E(X)$, $\mu$
\end{block}
\begin{block}{Formula}
\begin{align}
    E(X)=\mu=\sum k\pr{X=k}
\end{align}
where \pr{X} is the probability function of a distribution.
\end{block}
\end{frame}
\begin{frame}{}
\begin{block}{Median}
It is the value separating the higher half from the lower half of a data sample

Notation - $median(X)$
\end{block}
\begin{block}{Formula}
\begin{align}
    \int\limits_{lower limit}^{median(X)}f(X)d x=\frac{1}{2}=\int\limits_{median(X)}^{upper limit}f(X)d x
\end{align}
We can find $median(X)$ by solving this equation,

where $f(X)$ is the probability function.
\end{block}
\end{frame}
\begin{frame}{}
\begin{block}{Variance}
It is the expectation of the squared deviation of a random variable from its mean.

Notation - $V(X)$, $\sigma^2$
\end{block}
\begin{block}{Formula}
\begin{align}
    V(X)=\sigma^2=E((X-E(X))^2)=E(X^2)-(E(X))^2
\end{align}
\end{block}
\end{frame}
\begin{frame}{}
    \begin{block}{Exponential Distribution}
    It is a distribution in which events occur continuously and independently at a constant average rate
    
    Notation - $Exp(\lambda)$, where $\lambda$ is the rate parameter
    \end{block}
    \begin{block}{Probability function}
    \begin{align}
        f_X(x)=
        \begin{cases}
            \lambda e^{-\lambda x} & x\geq0\\
            0 & x<0
        \end{cases}
    \end{align}
    where $X$ is the random variable of the distribution.
    \end{block}
\end{frame}
\begin{frame}{Option 1}
    If $X$ has an exponential distribution then $median(X)<E(X)$
\end{frame}
\begin{frame}{Solution(Option 1)}
    Let $X$ has exponential distribution
    
    The expected value of $X \sim Exp(\lambda)$,
    \begin{align}
        E(X)=\int\limits_0^{\infty}x\lambda e^{-\lambda x}d x\\
        E(X)=\frac{1}{\lambda}
    \end{align}
    The median of $X \sim Exp(\lambda)$,
    \begin{align}
        \frac{1}{2}=\int\limits_0^{median(X)}\lambda e^{-\lambda x}d x\\
        median(X)=\frac{\ln{2}}{\lambda}
    \end{align}
\end{frame}
\begin{frame}{Solution(Option 1) Contd.}
    \begin{align}
        \ln{2}<1\\
        \frac{\ln{2}}{\lambda}<\frac{1}{\lambda}\\
         median(X)<E(X)
    \end{align}
    Hence, option $1$ is correct.
\end{frame}
\begin{frame}{}
    \begin{block}{Uniform Distribution}
     It is a distribution that describes an experiment where there is an arbitrary outcome that lies between certain bounds($[a,b]$)
    
    Notation - $U(a,b)$, where $[a,b]$ is the bounded region
    \end{block}
    \begin{block}{Probability function}
    \begin{align}
        f_X(k)=
        \begin{cases}
            \frac{1}{b-a} & a\leq x\leq b\\
            0 & x < a, x > b
        \end{cases}
    \end{align}
    where $X$ is the random variable of the distribution.
    \end{block}
\end{frame}
\begin{frame}{Option 2}
    If $X$ has a uniform distribution on an interval $[a,b]$, then $E(X)<median(X)$
\end{frame}
\begin{frame}{Solution(Option 2)}
    Let $X$ has Uniform distribution
    
    The expected value of $X \sim U(a,b)$,
    \begin{align}
        E(X)=\int\limits_a^{b}x\frac{1}{b-a}d x\\
        E(X)=\frac{1}{2}(a+b)
    \end{align}
    The median of $X \sim U(a,b)$,
    \begin{align}
        \frac{1}{2}=\int\limits_a^{median(X)}\frac{1}{b-a}d x\\
        median(X)=\frac{a+b}{2}
    \end{align}
\end{frame}
\begin{frame}{Solution(Option 2) Contd.}
    \begin{align}
        median(X)=E(X)
    \end{align}
    Hence, option $2$ is incorrect.
\end{frame}
\begin{frame}{}
    \begin{block}{Binomial Distribution}
     It is a distribution of the possible number of successful outcomes in a given number of trials in each of which there is the same probability of success.
    
    Notation - $B(n,p)$, where n = no. of trails, p = success parameter
    \end{block}
    \begin{block}{Probability function}
    \begin{align}
        f_X(k)=
        \begin{cases}
            {^n C_k}p^k(1-p)^{n-k} & 0\leq k\leq n\\
            0 & otherwise
        \end{cases}
    \end{align}
    where $X$ is the random variable of the distribution.
    \end{block}
\end{frame}
\begin{frame}{Option 3}
     $X$ has a Binomial distribution then $V(X)<E(X)$
\end{frame}
\begin{frame}{Solution(Option 3)}
    Let $X$ has Binomial distribution
    
    The expected value of $X \sim B(n,p)$,
    \begin{align}
        E(X)=\sum\limits_{k=0}^{n}k.{^n C_k}p^k(1-p)^{n-k}\\
        E(X)=n p
    \end{align}
    The variance of $X \sim B(n,p)$,
    \begin{align}
        V(X)=\sigma^2=E(X^2)-(E(X))^2\\
        V(X)=\sum\limits_{k=0}^n k^2.{^n C_k}p^k(1-p)^{n-k}\\
        V(X)=n p(1-p)
    \end{align}
\end{frame}
\begin{frame}{Solution(Option 3) Contd.}
    \begin{align}
        1-p\leq1\\
        n p(1-p)\leq n p\\
        V(X)\leq E(X)
    \end{align}
    Hence, option $3$ is incorrect.
\end{frame}
\begin{frame}{}
    \begin{block}{Normal Distribution}
     It is a distribution that is symmetric about the mean, showing that data near the mean are more frequent in occurrence than data far from the mean. 
     
     It is the more frequently used distribution.
    
    Notation - $N(\mu,\sigma^2)$, where $\mu = mean$, $\sigma^2 = variance$
    \end{block}
    \begin{block}{Probability function}
    \begin{align}
        f_X(k)=\frac{1}{\sigma\sqrt{2\pi}}e^{-(\frac{x-\mu}{2\sigma})^2}
    \end{align}
    where $X$ is the random variable of the distribution.
    \end{block}
\end{frame}
\begin{frame}{Option 4}
    $X$ has a normal distribution, then $E(X)<V(X)$
\end{frame}
\begin{frame}{Solution(Option 4)}
    Let $X$ has Normal distribution
    
    The expected value of $X \sim N(\mu,\sigma^2)$,
    \begin{align}
        E(X)=\mu
    \end{align}
    The variance of $X \sim N(\mu,\sigma^2)$,
    \begin{align}
        V(X)=\sigma^2
    \end{align}
    These are user defined values.
\end{frame}
\begin{frame}{Solution(Option 4) Contd.}
    So, $E(X)$ and $V(X)$ can take any value independent of each other.
    
    Hence, option $4$ is incorrect.
\end{frame}

\end{document}